\documentclass[luatex,fontsize=8pt,paper=b5,twoside]{jlreq}%
\usepackage{amsmath,amssymb}
\usepackage{booktabs,caption}
\usepackage{luwa-ul}
\usepackage[most]{tcolorbox}
\usepackage{luatexja-ruby}
\usepackage{modernruler}

\usepackage{listings}
\lstset{
    basicstyle=\ttfamily\small,
    keywordstyle=\color{blue},
    commentstyle=\color{gray},
    stringstyle=\color{red},
    breaklines=true,
    breakatwhitespace=false,  
    columns=flexible           
}

% You can omit these font settings.
\makeatletter
\RequirePackage[no-math]{fontspec}
\RequirePackage[no-math,match,scale=1]{luatexja-fontspec}
\RequirePackage[hiragino-pro,deluxe,expert]{luatexja-preset}
\setmainfont{HiraMinPro-W3}[BoldFont=HiraMinPro-W6]
\setmainjfont{HiraMinPro-W3}[BoldFont=HiraMinPro-W6]
\newfontfamily{\sfhira@pre}{HiraKakuPro-W3}[BoldFont=HiraKakuPro-W6]
\newjfontfamily{\sfhiraj@pre}{HiraKakuPro-W3}[BoldFont=HiraKakuPro-W6]
\newfontfamily{\mchira@pre}{HiraMinPro-W3}[BoldFont=HiraMinPro-W6]
\newjfontfamily{\mchiraj@pre}{HiraMinPro-W3}[BoldFont=HiraMinPro-W6]
\newfontfamily{\gthira@pre}{HiraKakuPro-W3}[BoldFont=HiraKakuPro-W6,FontFace={eb}{\shapedefault}{HiraKakuStd-W8}]
\newjfontfamily{\gthiraj@pre}{HiraKakuPro-W3}[BoldFont=HiraKakuPro-W6,FontFace={eb}{\shapedefault}{HiraKakuStd-W8}]
\newfontfamily{\mghira@pre}{HiraMaruPro-W4}
\newjfontfamily{\mghiraj@pre}{HiraMaruPro-W4}
\renewcommand{\sffamily}{\sfhira@pre\sfhiraj@pre}
\renewcommand{\mcfamily}{\mchira@pre\mchiraj@pre}
\renewcommand{\gtfamily}{\gthira@pre\gthiraj@pre}
\renewcommand{\mgfamily}{\mghira@pre\mghiraj@pre}
\makeatother
%%%


\usepackage{hyperref} 
\hypersetup{
  luatex, pdfencoding=auto, 
  colorlinks=true,
  linkcolor=black,     
  citecolor=black,     
  urlcolor=DeepSkyBlue3,      
  pdfborder={0 0 0}, 
}

\colorlet{grayLight}{white!80!black} 

\NewTCBListing{SourceCode}{ m m !o !O{DeepSkyBlue3} }{%
  enhanced, colback=black!70, colframe=Snow4,
  toptitle=-1mm, bottomtitle=-1mm,
  righttitle=-1mm, lefttitle=-1mm,
  arc=.5mm, 
  title={\tcbox[on line, arc=.5mm, boxsep=0pt, boxrule=0pt, top=1mm, bottom=0.8mm, left=2mm, right=2.2mm, colback=gray!80, coltext=white]{\raisebox{-0.1ex}{\vphantom{羅}\vphantom{j}#1}}},fonttitle=\gtfamily\footnotesize,boxrule=0.8pt,
  breakable,before upper={\color{white}},top=-0.5mm,bottom=-0.5mm,
  after title=\IfNoValueTF{#3}{}{{\hfill\tcbox[on line, arc=.5mm, boxsep=0pt, boxrule=0pt, top=1mm, bottom=0.8mm, left=2mm, right=2.2mm, colback=white!80!black, coltext=#4]{\raisebox{-0.1ex}{\vphantom{羅}\vphantom{j}#3}}}},
  listing only,
  listing options={
    language={#2},
    basicstyle=\ttfamily,
    keywordstyle=\ttfamily\color{white},
    stringstyle=\itshape\color{white},
    commentstyle=\small\gtfamily\color{DeepSkyBlue2},
    showspaces=false,showtabs=false,
    breaklines=true,breakindent=0pt,
    showstringspaces=false,
    columns=fullflexible,
    tabsize=2,
    numbers=left,numbersep=1.5pt,
    numberstyle=\scriptsize\gtfamily\color{gray},
  }
}

\NewTColorBox{OutPut}{ m !o !O{DeepSkyBlue3} }{%
  enhanced, colframe=Snow4,
  toptitle=-1mm, bottomtitle=-1mm,
  righttitle=-1mm, lefttitle=-1mm,
  arc=.5mm, colback=white, 
  title={\tcbox[on line, arc=.5mm, boxsep=0pt, boxrule=0pt, top=1mm, bottom=0.8mm, left=2mm, right=2.2mm, colback=gray!40, coltext=DeepSkyBlue3]{\raisebox{-0.1ex}{\vphantom{羅}\vphantom{j}#1}}},fonttitle=\gtfamily\footnotesize,boxrule=0.8pt,
  breakable,top=-0.5mm,bottom=-0.5mm,
  after title=\IfNoValueTF{#2}{}{{\hfill\tcbox[on line, arc=.5mm, boxsep=0pt, boxrule=0pt, top=1mm, bottom=0.8mm, left=2mm, right=2.2mm, colback=white!80!black, coltext=#3]{\raisebox{-0.1ex}{\vphantom{羅}\vphantom{j}#2}}}}, bottom=2mm, top=2mm, 
}


\title{\texttt{modernruler} Package Documentation}
\author{KKTeX}
\date{Version 1.0.0 (2025/12/28)}

\begin{document}
\begin{titlepage}
  \maketitle
\end{titlepage}
\newpage
\tableofcontents
\newpage

\section{Outline}
This package provides modern hrule (\verb|\mruleth|), vrule (\verb|\mruletv|), and \verb|\undernote| internally uses the two commands.

\begin{SourceCode}{Intput}{TeX}
  % normal rule
  \mruleth[height=1pt, width=5cm, color=blue]

  % dashed rule
  \mruleth[height=2pt, width=8cm, color=red, dash=true, dash-len=5pt, gap-len=3pt]

  % inline rule
  This is \mruletv[height=\zw, depth=1.5\zw, width=1pt, color=green] a modern vrule.

  % inline dashed rule
  This is \mruletv[height=2\zw, width=1.5pt, color=orange, dash=true, dash-len=2pt, gap-len=1pt] a dashed modern vrule.

  % undernote
  This is under note. This uses \undernote{these commands}{\texttt{\textbackslash mruleh} and \texttt{\textbackslash mrulev}} internally and enables make flexible outputs.
\end{SourceCode}
\begin{OutPut}{Output}
  % normal rule
  \mruleth[height=1pt, width=5cm, color=blue]

  % dashed rule
  \mruleth[height=2pt, width=8cm, color=red, dash=true, dash-len=5pt, gap-len=3pt]

  % inline rule
  This is \mruletv[height=\zw, depth=1.5\zw, width=1pt, color=green] a modern vrule.

  % inline dashed rule
  This is \mruletv[height=2\zw, width=1.5pt, color=orange, dash=true, dash-len=2pt, gap-len=1pt] a dashed modern vrule.

  % undernote
  This is under note. This uses \undernote{these commands}{\texttt{\textbackslash mruleh} and \texttt{\textbackslash mrulev}} internally and enables make flexible outputs.
\end{OutPut}

\section{Installation}
\begin{SourceCode}{Intput}{TeX}
  \usepackage[<options>]{modernruler}
\end{SourceCode}

Detailed information regarding the options will be provided in a later section, specifically during the explanation of the \verb|\undernote| command.

\section{Commands}
\subsection{\texttt{\textbackslash mruleth, \textbackslash mruletv}}
In the first place, \verb|\mruleth| and \verb|mruletv| are extended \verb|\hrule| and \verb|\vrule|. You can use them like this:

\begin{SourceCode}{Intput}{TeX}
  \mruleth[<keyvaloptions>]
  \mruletv[<keyvaloptions>]
\end{SourceCode}

These commands share the same keys.\bigskip

\begin{center}
  \begin{tabular}{llll}
    \toprule
    \textbf{Key} & \textbf{Type} & \textbf{Default} & \textbf{Description} \\ \midrule
    width & dim & 0pt & Width of the rule \\
    height & dim & 0pt & Height above the baseline \\
    depth & dim & 0pt & Depth below the baseline \\
    color & tl & black & Color of the rule \\
    gap-color & tl & white & Color of the gap segments \\
    dash & bool & false & Enables dashed line if true \\
    dash-len & dim & 3pt & Length of the dash segment \\
    gap-len & dim & 2.5pt & Length of the gap between dashes \\
    \bottomrule
  \end{tabular}
\end{center}\bigskip

The outputs are as follows:

\begin{SourceCode}{Intput}{TeX}
  \fboxsep=0pt\fboxrule=.1pt%

  % horizontal rule
  \fbox{\mruleth[height=1pt, width=55pt, color=blue]}
  \fbox{\mruleth[height=1pt, width=55pt, color=blue, dash=true]}
  \fbox{\mruleth[height=1pt, width=57pt, color=blue, dash=true]}
  \fbox{\mruleth[height=1pt, width=59pt, color=blue, dash=true]}

  % vertical rule
  \fbox{\mruletv[width=1pt, height=55pt, color=blue]}
  \fbox{\mruletv[width=1pt, height=55pt, color=blue, dash=true]}
  \fbox{\mruletv[width=1pt, height=57pt, color=blue, dash=true]}
  \fbox{\mruletv[width=1pt, height=59pt, color=blue, dash=true]}\bigskip

  % Horizontal: Alternating Black and Cyan segments
  \mruleth[width=120pt, height=2pt, dash=true, dash-len=6pt, gap-len=6pt, color=black, gap-color=cyan]\bigskip

  % Vertical: Warning Pattern (Yellow and Black)
  \mruletv[width=3pt, height=50pt, dash=true, dash-len=8pt, gap-len=4pt, color=yellow, gap-color=black]
\end{SourceCode}

\begin{OutPut}{Output}
  \fboxsep=0pt\fboxrule=.1pt%
  
  \fbox{\mruleth[height=1pt, width=55pt, color=blue]}
  \fbox{\mruleth[height=1pt, width=55pt, color=blue, dash=true]}
  \fbox{\mruleth[height=1pt, width=57pt, color=blue, dash=true]}
  \fbox{\mruleth[height=1pt, width=59pt, color=blue, dash=true]}

  \fbox{\mruletv[width=1pt, height=55pt, color=blue]}
  \fbox{\mruletv[width=1pt, height=55pt, color=blue, dash=true]}
  \fbox{\mruletv[width=1pt, height=57pt, color=blue, dash=true]}
  \fbox{\mruletv[width=1pt, height=59pt, color=blue, dash=true]}

  \bigskip

  % Horizontal: Alternating Black and Cyan segments
  \mruleth[width=120pt, height=2pt, dash=true, dash-len=6pt, gap-len=6pt, color=black, gap-color=cyan]

  \bigskip

  % Vertical: Warning Pattern (Yellow and Black)
  \mruletv[width=3pt, height=50pt, dash=true, dash-len=8pt, gap-len=4pt, color=yellow, gap-color=black]
\end{OutPut}

\subsection{\texttt{\textbackslash undernote}}
\begin{SourceCode}{Intput}{TeX}
  \undernote<star option>[<number of lines to shift the note downward>]{<The target phrase>}{<annotation>}[<style>]
\end{SourceCode}


\section{License}
Released under the MIT License.

\end{document}