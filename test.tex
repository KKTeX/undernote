\documentclass{jlreq}
\usepackage[parstyle=2,noteoverhang=3em]{modernruler}

\begin{document}
\fboxsep=0pt\fboxrule=.1pt%

あああああ\undernote{あああ}{あああああ}あああ

% 通常の実線
\mruleth[height=1pt, width=8cm, color=blue]

% 破線(dashオプション)
\mruleth[height=2pt, width=8cm, color=red, dash=true, dash-len=5pt, gap-len=3pt]

% 文中で垂直線を入れる
ここから垂直線 \mruletv[height=\zw, depth=1.5\zw, width=1pt, color=green] です。

% 破線の垂直線
点線の区切り \mruletv[height=2\zw, width=1.5pt, color=orange, dash=true, dash-len=2pt, gap-len=1pt] を作ります。
\end{document}